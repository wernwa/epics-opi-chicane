\input{header_script.tex}

\usepackage{amsmath}                % brauche ich um dir Formel zu umrahmen.
\usepackage{amsfonts}                % brauche ich für die Mengensymbole
\usepackage{amssymb}


\usepackage{listings}
\lstset{
   language=bash
  ,numbers=none
  ,basicstyle=\fontsize{11}{13}\selectfont\ttfamily
  ,showstringspaces=false
}
\usepackage{ngerman}

\begin{document}
\textit{\today}\\
\textit{Walter Werner}
\\
\\


\section*{$Z^+$-Lambda EPICs IOC Server}

This EPICs IOC Server is designed to control the voltage and current of the 10 $Z^+$-Lambda power-supplies that power 7 quadrupoles and 2 dipoles. The first power-supply is reserved for a relee function and the other 9 power the magnets. Setting the voltage to 20V of the relee power-supply means a positive sign to voltage and current of the magnets and setting the voltage to 0V means a negative sign.

\begin{eqnarray}
  20\text{V}&\leftarrow \text{POSITIVE}\notag\\ 
  0 \text{V}&\leftarrow \text{NEGATIVE} \notag
\end{eqnarray}

\subsection*{Records}

In EPICs therms records are variables that are visible on the LAN to the clients. To change and view the current and voltage of each $Z^+$-Lambda power-supplies threre are 20 variables defined:

\begin{lstlisting}[language=bash]
shicane:zps:relee:volt
shicane:zps:relee:curr
shicane:zps:2:volt
shicane:zps:2:curr
shicane:zps:3:volt
shicane:zps:3:curr
.
.
.
shicane:zps:10:volt
shicane:zps:10:curr
\end{lstlisting}

Per definition the values of each power supply are positive float numbers. In addition to these variables there are magnet records defined to represent also the sign:


\begin{lstlisting}[language=bash]
shicane:q1:volt
shicane:q1:curr
shicane:q2:volt
shicane:q2:curr
.
.
.
shicane:q7:volt
shicane:q7:curr

shicane:d1:volt
shicane:d1:curr
shicane:d2:volt
shicane:d2:curr
\end{lstlisting}

These magnet records are read only variables.

\subsection*{Starting/stopping the IOC-Server}

To start the server go first into the zps directory: 

\begin{lstlisting}[language=bash]
cd ~/epcis/ioc/zps
\end{lstlisting}

and start the server:

\begin{lstlisting}[language=bash]
./zps-ioc.py
\end{lstlisting}

Every EPICs server needs two ports to communicate with the clients. The Server itself resides in the python file \textbf{zps-ioc.py}. Since there can be many ioc-server on the same host, the ports have to be configured via the EPICS\_CA\_SERVER\_PORT and EPICS\_CA\_REPEATER\_PORT variables. This is done in the \textbf{zps-ioc.sh} shell script. Configure the ports in the shell script and start it like this:

\begin{lstlisting}[language=bash]
./zps-ioc.sh
\end{lstlisting}

Stopping the server use CTRL+C key combination. If the server does not react you can open another shell and run the stop script:

\begin{lstlisting}[language=bash]
./zps-ioc-stop.sh
\end{lstlisting}



\subsection*{EPICs command line}

EPICs provides several command line tools for a quick view and change of single Records. \textbf{caget}, \textbf{caput} and \textbf{camonitor} are sufficient for most puposes. To view the current value of the record shicane:zps:2:volt type:

\begin{lstlisting}[language=bash]
$ caget shicane:zps:2:volt
\end{lstlisting}

To write a new value 3.14 to the record shicane:zps:2:volt type:

\begin{lstlisting}[language=bash]
$ caput shicane:zps:2:volt 3.14
\end{lstlisting}

And to monitor this record for changes type:

\begin{lstlisting}[language=bash]
$ monitor shicane:zps:2:volt
\end{lstlisting}



\section*{Temperature EPICSs-IOC Server}

For the temperature monitoring for each magnet threre are record variables defined:

\begin{lstlisting}[language=bash]
shicane:q1:temp
shicane:q2:temp
shicane:q3:temp
shicane:q4:temp
shicane:q5:temp
shicane:q6:temp
shicane:q7:temp
shicane:d1:temp
shicane:d2:temp
\end{lstlisting}



\end{document}
